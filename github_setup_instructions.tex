\documentclass[11pt,obeyspaces]{article}
\usepackage{graphicx}
\usepackage{cite}
\usepackage{amssymb}
\usepackage{pdfsync}
\textwidth = 6.0 in
\textheight =  8 in
\oddsidemargin = 0 in
\evensidemargin = 0 in
\topmargin = 0.0 in
\headheight = 0.0 in
\headsep = 0 in
\parskip = 0.2in
\parindent = 0.0in
\newcommand{\la}{\lambda}
\newcommand{\lra}{$\longrightarrow$}
\newcommand{\ch}{$\bigtriangleup$}


\usepackage{fixltx2e}
\usepackage[colorlinks=true]{hyperref}
\usepackage{lmodern}
\usepackage[T1]{fontenc}
\usepackage{textcomp}
\usepackage[utf8]{inputenc}
\usepackage{microtype}
\usepackage{url}
\usepackage{amsmath,amssymb,amsfonts,amsthm,mathrsfs}
\usepackage{graphicx,enumerate,courier}
\usepackage{framed}
\usepackage[usenames,dvipsnames]{xcolor}

\begin{document}


\centerline{\bf \Huge Github Instructions}
\centerline{\sc For Pushing To Your First Repository}
\bigskip

\begin{framed}
\centerline{\sc some vocabulary:} 
{\bf Repository} (n): a directory on Github that contains a project \\
{\bf Fork} (v): To copy someone else's repository into your profile, so that it's available as one of your own repositories \\
{\bf Clone} (v): To make a copy of one of your repositories to your computer \\
{\bf Pull} (v): You submit a pull request if you think your code should be accepted into someone else's; you accept a pull request if you're in a position of authority and want to accept someone else's code\\
{\bf Master} (n): The trunk. The "official" working version of the project \\
{\bf Branch} (n): A copy of the master that you edit \\
{\bf Merge} (v): If your branch is compatible with the master, you can submit a pull request to combine the two (i.e., to merge)\\
{\bf Diff} (n): Short for differential, it's the differences between the branch and the master\\
{\bf Add} (v): To stage, or track, the files -- must be done before you commit, which must be done before you push \\
{\bf Commit} (v): To save a copy of the file, which you must do before you push \\
{\bf Push} (v): To make something from your computer available to manipulate on Github 
\end{framed}

\centerline{\sc to configure git:}
\centerline{\small blue is specific to me, black is general}
\path{> git config --global user.email "}\textcolor{blue}{\path{eliz.s.grace@gmail.com}}\path{"}\\
\path{> git config --global user.name} "\textcolor{blue}{\path{Elizabeth Grace}}\path{"}\\ \\

\centerline{\sc to push to a repository:}
\path{> cd ~}\\
Create a git folder "thesis" in home directory (must have same name as the git repository):\\
\path{> git clone} \textcolor{blue}{\path{https://github.com/elizabethgrace/thesis.git}} \\
Copy files to the git folder: \\
\path{> cd} \textcolor{blue}{\path{~/Documents/Thesis}} \\
\path{> cp} \textcolor{blue}{\path{-r IDL ~/thesis/}} \\
For the status: \\
\path{> git status} \\
To {\bf add}, which you need to do before you commit ({\bf .} means "all"): \\
\path{> git add .} \\ 
To {\bf commit} files, where {\bf -m} specifies that you're going to add a message flag and "initial commit" is the message here: \\
\path{> git commit -m "initial commit"} \\
To check that the files were committed:\\
\path{> git status}\\
To push the files to Github:\\
\path{> git push origin master}\\

\bigskip

\centerline{Shoutout to \href{https://github.com/annanuxoll}{\bf Anna Nuxoll} (clickable) for helping me out with all this!}








\end{document}
